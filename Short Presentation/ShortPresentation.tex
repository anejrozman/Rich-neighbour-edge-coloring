\documentclass[12pt,a4paper]{amsart}
% ukazi za delo s slovenscino -- izberi kodiranje, ki ti ustreza
\usepackage[slovene]{babel}
\usepackage[utf8]{inputenc}
%\usepackage[T1]{fontenc}
\usepackage{amsmath,amssymb,amsfonts}
\usepackage{url}
%\usepackage[normalem]{ulem}
\usepackage[dvipsnames,usenames]{color}
\usepackage{caption}
\usepackage{lipsum}
\usepackage{tikz}
\usepackage{xcolor}

\usetikzlibrary{graphs}
\usetikzlibrary{graphs.standard}

%\makeatletter
%\renewcommand\section{\@startsection{section}{1}%
%  \z@{.5\linespacing\@plus.7\linespacing}{.5\linespacing}%
%  {\normalfont\scshape\large\centering}}
%\renewcommand\subsection{\@startsection{subsection}{2}%
%  \z@{.5\linespacing\@plus.7\linespacing}{.5\linespacing}%
%  {\normalfont\scshape}}
%\renewcommand\subsubsection{\@startsection{subsubsection}{3}%
%  \z@{.5\linespacing\@plus.7\linespacing}{-.5em}%
%  {\normalfont\itshape}}
%\makeatother

% ne spreminjaj podatkov, ki vplivajo na obliko strani
\textwidth 15cm
\textheight 24cm
\oddsidemargin.5cm
\evensidemargin.5cm
\topmargin-5mm
\addtolength{\footskip}{10pt}
\pagestyle{plain}
\overfullrule=15pt % oznaci predlogo vrstico


% ukazi za matematicna okolja
\theoremstyle{definition} % tekst napisan pokoncno
\newtheorem{definicija}{Definition}[section]
\newtheorem{primer}[definicija]{Example}
\newtheorem{opomba}[definicija]{Remark}

\renewcommand\endprimer{\hfill$\diamondsuit$}

\theoremstyle{plain} % tekst napisan posevno
\newtheorem{lema}[definicija]{Lemma}
\newtheorem{izrek}[definicija]{Theorem}
\newtheorem{trditev}[definicija]{Statement}
\newtheorem{posledica}[definicija]{Corollary}
\newtheorem{conjecture}[definicija]{Conjecture}


% za stevilske mnozice uporabi naslednje simbole
\newcommand{\R}{\mathbb R}
\newcommand{\N}{\mathbb N}
\newcommand{\Z}{\mathbb Z}
\newcommand{\C}{\mathbb C}
\newcommand{\Q}{\mathbb Q}

% ukaz za slovarsko geslo
\newlength{\odstavek}
\setlength{\odstavek}{\parindent}
\newcommand{\geslo}[2]{\textbf{#1}\hspace*{3mm}\hangindent=\parindent\hangafter=1 #2}

% naslednje ukaze ustrezno popravi
\newcommand{\program}{Financial mathematics} % ime studijskega programa: Matematika/Finančna matematika
\newcommand{\imeavtorja}{Anej Rozman, Tanja Luštrek} % ime avtorja
\newcommand{\imementorja}{Assistant Professor Janoš Vidali} % akademski naziv in ime mentorja
\newcommand{\imesomentorja}{Professor Riste Škrekovski}
\newcommand{\naslovdela}{Rich-Neighbor Edge Colorings}
\newcommand{\letnica}{2023} %letnica diplome

\begin{document}

\thispagestyle{empty}
{\large
\noindent UNIVERSITY OF LJUBLJANA\\[1mm]
FACULTY OF MATHEMATICS AND PHYSICS\\[5mm]
\program\ -- 1st cycle}
\vfill

\begin{center}{\large
\imeavtorja\\[2mm]
{\bf \naslovdela}\\[10mm]
Term Paper in Finance Lab\\[2mm]
Short Presentation\\[1cm]
Advisers: \imementorja, \\ \imesomentorja\\[2mm]}
\end{center}
\vfill

{\large
Ljubljana, \letnica}
\pagebreak

\section{Introduction}

In this paper we set out to analyse an open conjecture in a modern graph theory problem known as rich-neighbor edge coloring.

\begin{definicija}
    In an edge coloring, an edge $e$ is called $rich$ if all edges adjacent to $e$ have different colors. An edge coloring is 
    called a $rich\text{-}neighbor \ edge \ coloring$ if every edge is adjacent to some rich edge.
\end{definicija}

\begin{definicija}
    $X'_{rn}(G)$ denotes the smallest number of colors for which there exists a rich-neighbor edge coloring.
\end{definicija}

\begin{conjecture}
    For every graph $G$ of maximum degree $\Delta$, $X'_{rn}(G) \leq 2\Delta - 1$ holds.\\
\end{conjecture}

\section{Plan}

Our plan is to create an integer program that ``proves'' the conjecture for (small) regular graphs of degree 4 or more (it finds a rich-neighbor edge coloring for every $k$-regular graph on $n$ verticies) and to make a random search algorythm for checking classes of graphs that are too large to be checked individually.

\subsection{Integer Programming}

Using SageMath we plan to construct an integer programming model that finds a rich-neighbor edge coloring for a given graph. Our interger program looks like this:\\

minimize $t$ \hfill \textcolor{gray}{we minimize the number of colors we need}\\

subject to $\forall e: \quad \sum_{i=1}^{k} x_{ei} = 1$ \hfill \textcolor{gray}{each edge is exactly one color}\\

\ \ \ \ \ \ \ \ \ \ \ \ \ \ $\forall i \ \forall u \ \forall v, w \sim u, v \neq u: \quad x_{uv, i} + x_{uw, i} \leq 1$\\[0.1mm]
\textcolor{white}{hihi} \hfill \textcolor{gray}{edges with the same vertex are a different color}\\

\ \ \ \ \ \ \ \ \ \ \ \ \ \ $\forall e \ \forall i: \quad x_{ei} \cdot i \leq t$ \hfill \textcolor{gray}{we use less or equal to $t$ colors}\\

\ \ \ \ \ \ \ \ \ \ \ \ \ \ $\forall i \ \forall uv \ \forall w \sim u, w \neq v \ \forall z \sim v, z \neq u, w: \quad x_{uw, i} + x_{vz, i} + y_{uv} \leq 2$ \hfill \\[0.1mm]
\textcolor{white}{hihi} \hfill \textcolor{gray}{$uv$ is a rich edge $\Leftrightarrow$ all adjacent edges are a different color}\\

\ \ \ \ \ \ \ \ \ \ \ \ \ \ $\forall e: \quad \sum_{f \sim e}y_f \geq 1$ \hfill \textcolor{gray}{every edge is adjacent to some rich edge}\\

\ \ \ \ \ \ \ \ \ \ \ \ \ \ $t \geq 2 \Delta - 1$ \hfill \textcolor{gray}{we use $\geq 2 \Delta - 1$ colors}\\

\ \ \ \ \ \ \ \ \ \ \ \ \ \ $\forall e: \quad 0 \leq y_{e} \leq 1$, $y_{e} \in \Z$\\

\ \ \ \ \ \ \ \ \ \ \ \ \ \ $\forall e \ \forall i: \quad 0 \leq x_{ei} \leq 1$, $x_{ei} \in \Z$,\\

where
\begin{align*}        x_{ei} = \begin{cases}
            1, \  \text{if edge $e$ has color $i$} \\
            0, \  \text{otherwise}
    \end{cases} & \text{and} & 
    y_{e} = \begin{cases}
        1, \  \text{if edge $e$ is rich} \\
        0, \  \text{otherwise.}
    \end{cases}
\end{align*}\\

We plan to look for appropriate graph colorings with $\geq 2 \Delta - 1$ colors because, if we find a coloring with $2 \Delta - 1$ colors then the smallest possible number of colors is smaller of equal to that and the conjecture holds for that graph. If we need more than $2 \Delta - 1$ colors then it does not hold. The computation is also a lot quicker this way.\\

Next we will determine at what point the computation of rich-neighbor edge coloring becomes too intense for this technique and we will switch to the random search algorythm.

\subsection{Random Search}

We will construct a random search algorythm that will check if the conjecture holds for a class of graphs that are too large to be fully checked. Since our linear program does not seem to be very effective (by observation), we will also take inspiration from an algorythm provided by Professor Riste Škrekovski. Graphs will be checked by simulated annealing.

\subsection{Comparison}
We will compare both approaches in terms of time efficiency.

\end{document}